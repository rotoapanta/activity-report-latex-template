% ------------------------------------------------------------------------------
% File: contenido.tex
% Description: Main content of the LaTeX document, covering the installation 
%              and configuration of ZeroTier on Raspberry Pi 5.
% Author: [Your Name]
% Last Modified: [Last Modification Date]
% ------------------------------------------------------------------------------

% ---------------------
% SECTION: Introduction
% ---------------------

\section{Introducción a ZeroTier}
ZeroTier es una solución de red privada virtual (VPN) que permite interconectar dispositivos de manera segura sin necesidad de configuraciones complejas en routers o firewalls. Es ideal para acceder de manera remota a una Raspberry Pi sin abrir puertos en el router.

% ------------------------
% SECTION: Prerequisites
% ------------------------

\section{Requisitos Previos}
Antes de comenzar, asegúrate de cumplir con los siguientes requisitos:
\begin{itemize}
    \item Una \textbf{Raspberry Pi 5} con \textbf{Raspberry Pi OS} instalado y acceso a Internet.
    \item Acceso a la terminal de la Raspberry Pi (puede ser local o vía SSH).
    \item Una cuenta en \href{https://my.zerotier.com}{ZeroTier}.
\end{itemize}

% ----------------------------------------
% SECTION: Installing ZeroTier on Raspberry Pi 5
% ----------------------------------------

\section{Instalación de ZeroTier en Raspberry Pi 5}
Sigue estos pasos para instalar ZeroTier en tu Raspberry Pi 5:

\subsection{Actualizar el sistema}
\begin{lstlisting}[language=bash]
sudo apt update && sudo apt upgrade -y
\end{lstlisting}

\subsection{Instalar ZeroTier}
\begin{lstlisting}[language=bash]
curl -s https://install.zerotier.com | sudo bash
\end{lstlisting}

\subsection{Verificar la instalación}
\begin{lstlisting}[language=bash]
sudo systemctl status zerotier-one
\end{lstlisting}
Si está activo, debería mostrar \texttt{active (running)}.

% ----------------------------------------
% SECTION: Configuring the ZeroTier Network
% ----------------------------------------

\section{Configuración de la Red ZeroTier}

\subsection{Crear una cuenta y red en ZeroTier}
\begin{itemize}
    \item Inicia sesión en \href{https://my.zerotier.com}{ZeroTier}.
    \item En el panel, haz clic en \textbf{"Create a Network"}.
    \item Copia el \textbf{Network ID} generado (ejemplo: \texttt{d5e5fb65374a3986}).
\end{itemize}

\subsection{Unir la Raspberry Pi a la red ZeroTier}
\begin{lstlisting}[language=bash]
sudo zerotier-cli join <Network_ID>
\end{lstlisting}

\subsection{Autorizar el dispositivo en ZeroTier}
\begin{itemize}
    \item Entra nuevamente al panel de ZeroTier y ve a la sección \textbf{Members}.
    \item Busca la Raspberry Pi y \textbf{autorízala} marcando la casilla correspondiente.
\end{itemize}

\subsection{Verificar la conexión}
\begin{lstlisting}[language=bash]
sudo zerotier-cli listnetworks
\end{lstlisting}
Debes ver un estado \texttt{OK} y una IP asignada en la red ZeroTier (ejemplo: \texttt{192.168.194.200}).

% ----------------------------------------
% SECTION: Connecting and Verifying ZeroTier Network
% ----------------------------------------

\section{Conexión y Verificación en la Red ZeroTier}
Si tienes otro dispositivo (como una PC o laptop) que deseas conectar a la misma red ZeroTier, sigue estos pasos:

\subsection{Instalar ZeroTier en una computadora}
Instala ZeroTier en tu computadora (Ubuntu, Windows o Mac).

\subsection{Unirse a la misma red}
\begin{lstlisting}[language=bash]
sudo zerotier-cli join <Network_ID>
\end{lstlisting}

\subsection{Autorizar la computadora}
Autoriza la computadora en el panel de ZeroTier.

\subsection{Verificar la conexión entre dispositivos}
\begin{lstlisting}[language=bash]
ping <IP_ZeroTier_PC>
ping <IP_ZeroTier_RPi>
\end{lstlisting}


% ----------------------------------------
% SECTION: ZeroTier Network Diagram
% ----------------------------------------

\section{Diagrama de la Red ZeroTier}
\begin{figure}[htbp]
    \centering
    \begin{tikzpicture}[node distance=2cm]
        % Dispositivos
        \node (rpi) [device, fill=blue!30] {Raspberry Pi};
        \node (pc) [device, right of=rpi, xshift=4cm, fill=purple!30] {PC/Laptop};
        \node (cloud) [network, below of=rpi, yshift=-2cm, fill=green!30] {Red ZeroTier};

        % Conexiones
        \draw [arrow, thick, blue!50!black] (rpi) -- (cloud);
        \draw [arrow, thick, blue!50!black] (pc) -- (cloud);
    \end{tikzpicture}
    \caption{Diagrama de la red ZeroTier conectando una Raspberry Pi y una PC.}
\end{figure}

% ----------------------------------------
% SECTION: Remote Access to the Raspberry Pi
% ----------------------------------------

\section{Acceso Remoto a la Raspberry Pi}
Puedes conectarte a tu Raspberry Pi de forma remota a través de SSH:
\begin{lstlisting}[language=bash]
ssh pi@<IP_ZeroTier_RPi>
\end{lstlisting}

% ----------------------------------------
% SECTION: Troubleshooting
% ----------------------------------------

\section{Solución de Problemas}

\subsection{No se puede hacer ping entre dispositivos}
\begin{itemize}
    \item Verifica que el \textbf{firewall} de la Raspberry Pi permite tráfico en \textbf{ZeroTier}:
    \begin{lstlisting}[language=bash]
sudo ufw allow 22/tcp
sudo ufw enable
    \end{lstlisting}
    \item Asegúrate de que ambos dispositivos estén en la misma red y hayan sido autorizados.
\end{itemize}

\subsection{No se puede conectar por SSH}
\begin{itemize}
    \item Asegúrate de que el servicio SSH está corriendo en la Raspberry Pi:
    \begin{lstlisting}[language=bash]
sudo systemctl start ssh
    \end{lstlisting}
    
    \item Verifica que el puerto 22 está abierto en el firewall:
    \begin{lstlisting}[language=bash]
sudo ufw allow 22/tcp
    \end{lstlisting}
\end{itemize}

% ----------------------------------------
% SECTION: Conclusion
% ----------------------------------------

\section{Conclusión}
ZeroTier es una excelente solución para acceder de manera remota a tu Raspberry Pi sin complicaciones de redes y configuración de routers. Siguiendo este manual, puedes instalar, configurar y utilizar ZeroTier en tu Raspberry Pi 5 de manera efectiva.
